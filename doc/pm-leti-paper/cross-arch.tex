\section{Кросс-архитектурная миграция процессов}

Процесс является контейнером ресурсов ОС, причем состояние многих ресурсов не зависит от конкретной аппаратной архитектуры, поэтому в рамках исследования мы хотим рассмотреть возможность кросс-архитектурной миграции процессов, т. е. миграции процессов, в условиях различных аппаратных архитектур исходного и целевого узла миграции.

К ресурсам, которые не зависят от аппаратной архитектуры можно отнести сетевые соединения, так как современные распространенные сетевые протоколы спроектированны, чтобы быть независимыми от аппаратной платформы, и ресурсы файловой системы. К ресурсам зависимым от аппаратной платформы можно отести память, так как на различных системах могут существовать различные ограничения на выравнивание данных в памяти и доступ к невыровненым данным, а так же различный порядок байт, и состояние процессора, так как разные процессоры обладают разным набором регистров и инструкций.

Для преодоления различий аппаратных платформ разработан большой набор средств, от аппаратных~\cite{X86ARMBHI,ISAVIRT}, до программных. К программным средствам относятся различные виртуальные машины, например, распространненые Java и .NET виртуальные машины позволяют создавать кросс-платформенные программы, независимые от конкретной аппатаной платформы и даже ОС, другой пример - эмулятор QEMU, который позволяет исполнять нативные приложения разработанные для одной архитектуры, на другой.
