\section{Ограничения разработанного подхода}

Рассмотренный подход к реализации кросс-архитектурной миграции обладает рядом ограничений. Первое ограничение связано с набором переносимых ресурсов, так как переносятся только память и состояние процессора это весьма ограничивает класс переносимых процессов. Процессы кроме памяти взаимодействуют с файловой системой, с сетью и другими процессами. Для полноценной миграции процессов необходимо поддержать перенос состояния файловых дескрипторов и сетевого стека.

Следующим ограничением является использование памяти процессом. Память процесса в ОС Linux используется неравномерно, т. е. используемый объем памяти процесса может быть невелик, но часть памяти будет находиться в нижних адресах, а часть в верхних. Это делает невозможным создание копии адресного пространства большого класса процессов.

Наконец последнее препятсвие для широкого использования описанного подхода является потеря производительности. В некоторых тестах потеря производительности достигала двух порядков, что является неприемлемым для промышеленного использования кросс-архитектурной миграции процессов.
