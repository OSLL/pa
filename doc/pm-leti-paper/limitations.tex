\section{Ограничения разработанного подхода}

Рассмотренный подход к реализации кросс-архитектурной миграции обладает рядом ограничений. В данной работе мы намеренно ограничились самым базовым набором ресурсов: памятью и состоянием процессора. Но для полноценной миграции процессов необходимо обеспечить перенос многих других ресурсов, например, файловых дескрипторов или сокетов, без которых миграция процессов не имеет прикладной ценности.

Как было показано ранее память процсса используется неравномерно, это приводит нас к ограничению адресного пространства. Так как qemu требует восстановления памяти процесса без относительных смещений, в адресном пространстве qemu просто не хватает места для восстановления типичного процесса Linux.

\begin{table}[h]
  \begin{center}
    \begin{tabularx}{\linewidth}{|c|c|c|}
      \hline
      type                      & overhead & native \\
      \hline
      mem io:                   & 0.009    & 0.007  \\
      int division:             & 0.129    & 0.008  \\
      file io:                  & 1.330    & 0.424  \\
      \hline
      \multicolumn{3}{|X|}{\emph{overhead} - накладные расходы вносимые динамической трансляцией.} \\
      \multicolumn{3}{|X|}{\emph{native} - время исполнения без динамической трансляции.} \\
      \hline
    \end{tabularx}
  \end{center}
  \caption{\label{tab:perf}Результаты измерения производительности.}
\end{table}

Кроме этого использование динамического транслятора qemu приводит к серьезной потере производительности. В некоторых тестах потеря производительности достигала двух порядков (см. табл.~\ref{tab:perf}), что является неприемлимым для промышленного использования кросс-архитектурной миграции процессов.
