\section{Миграция процессов}

Миграция процессов - это перенос процессов между вычислительными узлами во время исполнения. Переносимое состояние включает в себя адресное пространтсво процесса, состояние процессора, внешние связи процесса и другие ресурсы. Эта технология позволяет использовать более гибкие механизмы распределения нагрузки, повысить устойчивость системы к ошибкам, упростить администрирование.

Не смотря на эти приемущества до недавнего времени технологии миграции процессов не получали большого распространения. Одной из причин такого состояния является сложность реализации миграции процессов в современных операционных системах, изначально разрабатываемых без учета миграции процессов. В качестве примера приведем проект linux-cr, по реализации механизма checkpoint/restart в ядре Linux. Исходные коды так и не были приняты в ядро Linux из-за слишком большого объема и сложности~\cite{CRCOMPLICATED}.

Однако с современным уровнем развития технологий виртуализации, а так же с учетом предыдущего опыта, реализация миграции процессов становится более чем реальной~\cite{CRIUPROPSAL}. Например, в проекте CRIU учли ошибки предшественника (linux-cr) и реализовали механизм checkpoint/restart для Linux преимущественно в пространстве пользователя с минимальной поддержкой со стороны ядра ОС.
