\section{Дальнейшие исследования}

Рассмотренный подход показывает, что реализация кросс-архитектурной миграции процессов возможна, однако разработка промышленной реализации требует решения ряда задач.

Qemu позиционируется как быстрый и переносимый транслятор, однако он почти не поддерживает оптимизаций над внутренним представлением~\cite{TCG}. Поэтому интересным направлением является использование более продвинутого бинарного траснлятора поддердивающего большее количество оптимизаций над промежуточным представлением. В качестве основы для такого транслятора можно использовать LLVM~\cite{LLVM}, который поддержвает большое число аппаратных платформ и различных оптимизаций.

Процессы в соременных ОС работают в сложном окружении, разделяемые библиотеки являются важной частью такого окружения. У популярных библиотек существуют версии под различные архитектуры (например, стандартная библиотека языка C), это открывает возможность для оптимизаций. Вместо трансляции кода динамической библиотеки, как это сейчас делает qemu, можно использовать нативную для целевой платформы версию библиотеки, собранную с помощью оптимизирующего компилятора.

Ограничения накладываемые неравномерным использованием адресного пространства процесса серьезно снижают применимость описанного подхода. Для преодоления этих ограничений можно использовать поддержку со стороны ядра операционной системы, для управления адресным пространством процесса.
